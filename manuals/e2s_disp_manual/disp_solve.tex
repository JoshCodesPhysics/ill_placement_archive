\section{Disp\_solve}
\subsection{What is Disp\_solve?}
Disp\_solve is a script that generates a number of atomic displacement cell
files for a crystal compound. These displaced unit cell atoms result from
enacting a range of uniform electric fields on a system. It takes input files
from a CRYSTAL~\footnote{Density functional quantum ab-initio program.}
simulation to generate the necessary matrices for this displacement
calculation.

\subsection{How to Use Disp\_solve}
\subsubsection{Installing python 3 and necessary modules}
A \textbf{python 3} compiler must be installed (see section $2.1.1$ for
information on how to install python 3 or an associated IDE~\footnote{Integrated development environment}). If an error message occurs that
mentions a missing module, execute the command

`\texttt{pip install <module\_name>}'

in the terminal. The following python modules are required to compile
the script:
  \begin{itemize} \itemsep -0.5ex
 	\item numpy
 	\item sys
 	\item itertools
 	\item copy
 	\item os.path
 	\item tabulate
 	\item ast
 	\item pathlib
 	\item json
 \end{itemize}

Most if not all of these modules should be native to your python 3 package. You can check if you have these by executing the command:

`\texttt{pip list | grep <module\_name\_you\_want\_to\_check>}'

or by running the script and interpreting the error for missing modules.

\subsubsection{Required Files}
The following files are required by the python script:
\begin{itemize}  
  \item (Compulsory) the CRYSTAL ouput file of a phonon calculation of your system, including IR intensities (Born charges) and Hessian matrix printed in Cartesian coordinates.
  \item (Compulsory) the system.cell input file of the env code corresponding to the undistorted system.
  \item (Optional) the disp\_solve command line input file. If absent prompts will ask you for the input data.  
\end{itemize}

\subsubsection{Executing the Script and Input File Format}
Once you have made sure all of these files are present, you have two options: 
\begin{itemize}
	\item `\texttt{python3 disp\_solve.py}' (with no input file: the code will prompt you for the inputs).
	\item `\texttt{python3 disp\_solve.py <input\_file\_path>}' Passing input file as a command line argument.
\end{itemize}

The prompt option allows you to enter the information directly into the terminal. You will receive the following prompts:

\begin{itemize}
	\item Unit cell generator parameter (direct,charge,auto).
	
	Enter \texttt{direct} if you want the unit cell irreducible atom information, coordinates and charge data to all come from initial cell file.
	
	Enter \texttt{charge} or \texttt{auto} if you want the coordinates and irreducible atom groups to come from crystal output file.
	
	\texttt{auto} automatically generates charge data from initial cell file, whereas \texttt{charge} allows for manual input.
	
	\item Name/full path of crystal output file:
	
	Input the name of the crystal output if it is local to the script directory, or the full path of the file if it is not.
	
	\item Name/full path of initial cell file containing the unit cell and charge data:
	
	Same again, name if local, full path if non-local.
	
	\item start,stop,step separated by commas for $E_a$ ($E_b$, $E_c$ are done sequentially after).
	
	Simply input starting electric field value, final value and size of step as a tuple.
	
	\item If \texttt{charge} is chosen, float charge values will then be prompted
\end{itemize}

The alternative is writing an input text file that contains the above information in a certain format. Below is an example of an input file format:

\begin{lstlisting}
	crystal_file = ht.frequence.B1PW_PtBs.loto.out
	cell_init = ymno3.cell
	ea = [0,0.5,0.1]
	eb = [0,0.2,0.05]
	ec = [0,1,0.2]
	unit_source = charge
	charge_dict = {"Y":3.0, "MN":3.0, "O1":-2.0, "O2":-2.0}
	
\end{lstlisting}

We have the following rules:

\begin{itemize}
	\item Each line element is separated by a space i.e. there is a space either side of the `$=$' sign.
	\item \texttt{crystal\_file} is the variable assigned to the relative/full path directory of the crystal output file.
	\item \texttt{cell\_init} is assigned to the relative/full path directory of the initial cell file.
	\item The \texttt{ea, eb, ec} lists must be in the format [start,stop,step] where start is the initial value, stop is the final value and step is the increment between values.
	\item \texttt{unit\_source} is the variable assigned to the unit cell generator parameter (charge,auto,direct).
	\item If \texttt{charge} is passed, a dictionary in python format must assign each atom to a charge value. The general format is \texttt{\{"key":value\}}. See section 2.3 for more examples.
	
\end{itemize}

Once this input file has been written, one can execute the script with this input file passed as an argument on the command line as above. 

\subsubsection{Output files}

\begin{itemize}
	\item Output cell file is named after the initial cell file with an added (Ea,Eb,Ec) component label. If the files are called locally, output cell file grid will be generated and stored in a new folder inside the script directory that is named after the array parameters for a,b,c.
	\item If the inputs files are called from full path, the output cell files will be generated and stored in a new folder inside either the directory of the cell file or the output file (specified by the command line).
\end{itemize}



%%% Local Variables:
%%% mode: latex
%%% TeX-master: driver_manual
%%% End:


